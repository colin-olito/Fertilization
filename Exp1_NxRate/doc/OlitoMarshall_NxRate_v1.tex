%%%%%%%%%%%%%%%%%%%%%%%%%%%%%
%Preamble
\documentclass{article}


%Dependencies
\usepackage[left]{lineno}
\usepackage{titlesec}
\usepackage{amsmath}
\usepackage{amsfonts}
\usepackage{amssymb}
\usepackage{color,soul}
\usepackage{ogonek}
\usepackage[hidelinks]{hyperref}

% Packages from Am. Nat. Template
\usepackage[sc]{mathpazo} %Like Palatino with extensive math support
\usepackage{fullpage}
\usepackage[authoryear,sectionbib,sort]{natbib}
\linespread{1.7}
\usepackage{lineno}

% Graphics
\usepackage{graphicx}
\graphicspath{{../output/figs}.pdf}
% New commands: fonts
%\newcommand{\code}{\fontfamily{pcr}\selectfont}
%\newcommand*\chem[1]{\ensuremath{\mathrm{#1}}}
\newcommand\numberthis{\addtocounter{equation}{1}\tag{\theequation}}
 

% New math operators
\DeclareMathOperator{\Var}{Var}


%%%%%%%%%%%%%%%%%%%%%%%%%%%%%
% Title Page

\title{Overdetermined sperm release phenologies in broadcast spawners: selection during sperm dispersal and sperm competition favour similar phenotypes}
\author{Colin Olito$^{\ast}$ and Dustin J.~Marshall}
\date{\today}

\begin{document}
\maketitle


\noindent{} Centre for Geometric Biology, School of Biological Sciences, Monash University, Victoria 3800, Australia.
\noindent{} $^\ast$ Corresponding author e-mail: colin.olito@gmail.com

\bigskip

\noindent{} \textit{Keywords}: xxx, ..., xxx

\bigskip

\noindent{} \textit{Manuscript type}: Brief Communication

\bigskip


% Set line number options
\linenumbers
\modulolinenumbers[1]
\renewcommand\linenumberfont{\normalfont\small}

%%%%%%%%%%%%%%%%%%%%%%%%%%%%%
% Main Text

\newpage{}
\section*{Abstract}

\noindent{} \hl{...}

\section*{Introduction}

% Need to be deft, both here and in the discussion, in explaining why we do not do this in a sperm-competitive context, despite the fact that ALL the theory is about SC. The play is: absence of sperm competition is the baseline, selection from environment is expected to be VERY strong for broadcasters, and our results suggest that SC doesn't change the predictions that much.

Our expectations and predictions regarding the outcome of sexual selection are heavily influenced by the idea that males compete strongly for the fertilization of females' eggs (\citealt{Bateman1948, Parker1972, Parker1982}). Indeed, sperm competition theory (SCT) represents the largest and most influential body of theory explaining observed patterns of male reproductive investment and allocation (\citealt{Parker1972, Parker1982,Wedell2002}). Classic predictions from SCT include increased ejaculate investment with competition among species, and facultative adjustment of ejaculate allocation within species to better compete for fertilizations (reviewed in \citealt{Wedell2002}). While the consequences of sperm competition for the evolution of male reproductive phenotypes have been well studied in internally fertilizing taxa, and some externally fertilizing fish species, it is unclear how relevant these predictions are for the archetypal external fertilizer: broadcast spawners (\citealt{BodeMarshall2007, Olito2015}). This is despite the fact that broadcast spawning represents the ancestral mode of reproduction (\citealt{RouFitz1994}), excludes complicating factors common in internal fertilizers such as cryptic female choice, and have long been used as a model for SCT and the evolution of anisogamy (e.g., \citealt{Parker1972, Parker1982}). Furthermore, broadcast spawners are a taxonomically diverse group, representing more than 50\% of global marine invertebrate biodiversity (\citealt{MonroMarshall2015}).

Superficially, broadcast spawning species appear well suited to predictions from sperm competition theory. Gametes are released into the water column by multiple individuals, often in dense aggregations (\citealt{McEuan1988,Levitan1998,Levitan2002, Levitan2004, Marshall2002}). Consequently, polyandry is common, and sperm competition can be intense for many broadcast spawning species (\citealt{Levitan2002, Levitan2004}). Indeed, recent SCT models incorporating realistic assumptions for broadcast spawners make a variety of predictions that are consistent with well documented empirical patterns, but which are inconsistent with SCT predictions tailored for internal fertilizers (\citealt{BodeMarshall2007, Olito2015, Olito2017}). Many species exhibit surprisingly long population spawning phenologies characterized by slow gamete release rates by individuals (\citealt{McEuan1988,MarshallBolton2007}). Protandry is common among broadcast spawners, with males releasing sperm earlier for for longer than females do eggs (\citealt{McEuan1988, LotterhosLevitan2011, Levitan2005, MarshallBolton2007}). In some species, large males do not necessarily release more sperm than small males, despite having the resources available to do so (\citealt{Styan2003}). Moreover, broadcast spawners often show massive investment in gonad tissue relative to even the most extravagant internal fertilizers (\hl{NEED TO PUT A NUMBER AND SOME CITATIONS ON THIS}). 

Given the influence of SCT, it is unsurprising that previous theoretical and empirical studies have emphasized the fertilization success of different male broadcast spawning strategies in a sperm competitive context. However, observed broadcast spawning strategies may also be a consequence of strong selection on gametes after they are released into the water column -- a major difference between broadcast spawners and internal ferilizers. For broadcast spawners, individual fertilization success is determined by the fertilization kinetics of sperm-egg interactions, individuals' spawning phenologies, hydrodynamic conditions, and the resulting local gamete concentrations (\citealt{DennyShib1989, Levitan1991, Levitan2002, Marshall2002, Levitan2004}). Crucially, males are able to directly manipulate local sperm concentrations via their sperm release phenologies. Thus, male spawning phenologies represent an important target of selection during the process of gamete dispersal. The critical question is: what are the fitness consequences of different male spawning phenologies? 

Very few studies have examined selection on male reproductive phenotypes from the external environment outside of a sperm competitive context (\citealt{Levitan2005, MarshallBolton2007}; reviewed in \citealt{LotterhosLevitan2011}). This is a crucial omission because the absence of sperm competition represents our baseline expectation for what an adaptive spawning strategy should be for broadcast spawners, and because selection on gametes from the external environment is expected to be very strong (cite...\citealt{DennyShib1989,Levitan2005,LotterhosLevitan2011}). For example, using a flow-through flume experiment \citet{BodeMarshall2007} found that males releasing sperm slowly enjoyed higher fertilization success at downstream egg patches than did fast-releasing males. This result suggests that slow sperm relase rates that are predicted for broadcast spawners by SCT may also emerge solely from selection during sperm dispersal. Here we aim to explore the consequences of selection during sperm dispersal on more diverse reproductive phenotypes in order to identify the conditions under which selection arising from the process of gamete dispersal, and selection arising from sperm competition, are expected to favour the same or different phenotypes. We focus on the reproductive success in the absence of sperm competition associated with two crucial traits contributing to male reproductive phentoypes in broadcast spawners: male energetic investment in ejaculates, and the rate of sperm release. 

To address this question empirically, we conducted two flow-through flume experiments to simulate different male reproductive phenotypes using gametes collected form the broadcast spawning marine invertebrate \textit{Galeolaria saespitosa}. We experimentally manipulated the amount, and the rate at which sperm was released, and analyzed the resulting fertilization success of two downstream egg patches. We found that the sperm release rate that resulted in higer fertilization success depended critically on the total amount of sperm being released. We also found that the per-gamete fertilization rate for males decreases dramatically with the total amount of sperm released - highlighting the massive investment required by males to overcome sperm wastage. Our results indicate that if males can facultatively adjust the size of their ejaculates, they should always release very little sperm, and slowly. Crucially, this result is the same as would be predicted under sperm competition, suggesting that well documented empirical patterns, and especially slow sperm release rates, are expected to evolve whether or not males experience sperm competition.


\section*{Methods}

	\subsection*{Study species}
	We studied spawning phenologies and the fertilization success of different male spawning strategies using \textit{Galeolaria caespitosa} Savigny 1818. \textit{G.~caespitosa} is a dioecious sessile polychaete worm common in the intertidal zone throughout southeast Australia. All individuals were collected from pier pilings at the Royal Brighton Yacht Squadron, Port Philip Bay, Victoria. For flume experiments, we used established protocols to collect gametes (e.g., \citealt{MarshallEvans2005a, MarshallEvans2005b}). \textit{G.~caespitosa} sperm become active immediately upon release, and can remain viable for several hours after spawning (\citealt{Kupriyanova2013}). To minimize any effects of sperm aging on egg fertilization success, we ensured that all gametes had been freshly spawned no more than $45$~min before being used in experiments.

	\subsection*{Lab spawnings}
	To estimate the duration of individual spawning phenologies, we documented induced, non-traumatic, spawnings in a laboratory setting. A large \textit{G.~caespitosa} colony fragment was collected on \date{16/11/16}, and stored overnight in a controlled temperature room at the ambient temperature of Port Philip Bay ($17.5^o$C). The large colony fragment was then sub-divided into small fragments of approx.~$50-100$ individuals that were then induced to spawn by heat shock in a shallow sea-water bath. Individual spawning males were marked and monitored and the duration of their spawning timed. Water in the bath was gently agitated, and spawning individuals were periodically flushed with seawater using a 1cc syringe. Individual males' spawning phenology was considered complete when no more ejaculate could be seen leaking from the tube with gentle flushing. Approximately $8-10$ small colony fragments were monitored at once in an experimental run, and we performed $5$ experimental runs resulting in $26$ documented individual spawning phenologies. 

	\subsection*{Flume experiments}
	We performed two flume experiments (Exp.~1 and Exp.~2 hereafter) to examine the effect of the amount and rate of sperm released on fertilization success in simulated spawning events. The flume was plexiglass, $240 \times 96 \times 10$ cm (L$\times$W$\times$D), divided into $6$ parallel lanes, each $15$ cm wide. Filtered seawater sourced from the Mornington Peninsula, Port Philip Bay, was gravity fed from a $20$ L constant head tank into a common head for the flume which in turn fed all $6$ lanes. Water was not recycled after flowing through the flume. During experimental runs, the flume was filled to a depth of $5$ cm, with a constant flow rate of $\approx 1$ cm per second across all lanes. Laminar flow in each lane was maintained by $10$ cm collimators made of drinking straws located $25$ cm from the head of each lane (\citealt{YundMeidel2003}). \textit{G.~caespitosa} colony fragments for both flume experiments were collected from the Royal Brighton Yacht Squadron between 9/10/2015 and 20/10/2015 for Exp.~1, and between 11/04/2016 and 22/04/2016 for Exp.~2. All flume experiments were conducted in a Controlled Temperature room maintained at $17^o$ C at the Jock Marshall Reserve Environmental Laboratories of the Monash Centre for Geometric Biology.

	In Exp.~1, we examined the effect of releasing different amounts of sperm on fertilization success at a patch of eggs located downstream from the sperm release point. Prior to running the flume, we collected gametes from $12$ males and $6$ females. All collected eggs were pooled together and mixed thoroughly. Eggs were then divided into $6$ aliquots that were gently released onto the bottom of the flume, in the center of each lane, $5$ cm downstream from the collimators ($25$ cm downstream from the sperm release point). \textit{G.~caespitosa} eggs are negatively buoyant, and the flow rate in the flume was low enough that eggs were not washed away. 

	All collected sperm was combined into a single pooled ejaculate, and thoroughly mixed. $0.1$ mL of the pooled ejaculate was set aside for sperm enumeration using a haemocytometer. The remaining ejaculate was diluted by a factor of $10$, and then sub-divided into $6$ aliquots with different volumes corresponding to $N = 0.5$, $1$, $1.5$, $2$, $2.5$, and $3$ times the average amount of sperm released by the individual males used in that experimental run. Thus, the specific amount of sperm released for a given level of $N$ differed among experimental runs. This was done intentionally in order to harness natural variation in the average amount of sperm collected from males for each experimental run, and sample a broader range of the amount of sperm released, while retaining the same ratios of relative male investment in sperm. Dilutions were performed immediately before initiating an experimental run to minimize sperm aging effects (i.e., after priming the flume, placing eggs, and starting the flow of water). Lane assignments for each level of $N$, and the order in which lanes were run, were randomized for each experimental run to account for any lane-specific, or lane order, effects on fertlization success. 

	During an experimental run, sperm was released from a syringe $15$ cm from the head of each lane ($10$ cm upstream from the collimators), in the center of each lane, immeditely above the bottom of the flume. Sperm was released over a period of $15$ seconds, and care was taken to release sperm at a steady rate. After all sperm was released, the flume was left to run uninterrupted for $10$ min before collecting eggs. Eggs were then washed with fresh filtered seawater, and left to develop in polyethelene test tubes for an additional $2$ hours. After incubation, $100$ eggs from each egg patch (lane) were then examined under an inverted compound microscope, and scored as either unfertilized, fertilized, or polyspermic, based on the presence/absence of raised egg envelopes, regular, or irregular, patterns of cell division. The flume was drained completely and washed with fresh water between each experimental run. We performed $8$ experimental runs, yielding a total of $n=48$ observations.

	In Exp.~2, we examined the effect of a factorial cross between the amount of sperm released, and the rate at which it was released, on fertilization success at two patches of eggs located downstream from the sperm release point. Wherever possible, we used identical protocols to Exp.~1, and describe only the relevant differences here. Eggs from $10$ females were collected, combined, and placed on the bottom of the flume in each lane in two patches located $25$ cm and $75$ cm downstream from the sperm release point. Sperm from $12$ males was collected, pooled, diluted by $10^{-1}$, and then sub-divided into $6$ aliquots with three different volumes corresponding to $N = 1$, $2$, and $3$ times the average amount of sperm released by the individual males used in that experimental run ($2$ aliquots per level of $N$). For each level of $N$, one replicate aliquot was released at a `fast' rate, corresponding to a single pulse lasting $10$ seconds. The second replicate aliquot was released at a `slow' rate, corresponding corresponding to $4$ pulses, each lasting $15$ seconds, spread out over $4$ minutes. Lane assignments and the order of release were again randomized. We performed $10$ experimental runs, yielding a total of $n=120$ observations ($10$ runs $\times~6$ lanes $\times~2$ egg patches $= 120$).

	\subsection*{Analyses}
	For both flume experiments, we modeled the number of fertilized eggs ($n$) in a given egg patch as a binomial response variable, where the number of eggs counted in each egg patch ($E$) corresponds to the number of bernoulli trials. We analyzed fertilization success using logistic regression in a Bayesian Linear Mixed Effects Regression framework, with the basic model structure 

\begin{align*}
	n_i   &\sim Bin\Big(E_i~|~logit^{-1}(\mu_i)\Big), \numberthis\\
	\mu_i &= \beta \mathbf{X} + \gamma \mathbf{Z} + \epsilon,
\end{align*}
\noindent{} with priors
\begin{align*}
	\beta           &\sim N(0,3) \\
	\gamma          &\sim N(0,\sigma_{\gamma}) \\
	\sigma_{\gamma} &\sim \mathit{half} \text{--}\textit{Cauchy}(0,1),
\end{align*}

	\noindent where $\beta$ and $\gamma$ are vectors of regression coefficients for `fixed' and `random' effects, $\mathbf{X}$ and $\mathbf{Z}$ are model matrices for the linear predictors, and $\epsilon$ is the residual error. Although there is little distinction between `fixed' and `random' effects in Bayesian analyses, this model specification allows for flexible heirarchical structure, and direct sampling of the posterior distributions for all parameters rather than maximum likelihood estimates. The number of sperm released, $Sperm_i$, was treated as a continuous predictor variable for all analyses. To avoid estimating very small regression coefficients (on the order of $1 \times 10^{-9}$), and to simplify interpretation of parameter estimates, the number of sperm released was Z-transformed prior to analyses. Thus, the units for $Sperm_i$ coefficients are in standard deviations of the empirical distribution of the number of sperm released in each experiment ($7.9 \times 10^{7}$ for Exp.~1; $8.9 \times 10^{7}$ for Exp.~2). 

	For Exp.~1, we fit a simple logistic regression of fertilization success regressed on the number of sperm released ($\mathbf{X} = Sperm_i$), with experimental run as a `random' grouping variable. We then analyzed the full set of nested models arising from the $\mathbf{Z} = Run \times Sperm_i$ interaction (with all lower-order interactions included). For Exp.~2, we treated the rate of sperm release ($Rate$, with two levels: 'Fast' or `Slow'), and the position of the downstream egg patches ($EggPos$, with two levels: '25 cm' or `75 cm') as categorical variables, and modeled fertilization success using a linear model of the three way interaction between of the amount of sperm released, the rate it was released, and the downstream distance to the egg patches ($ \mathbf{X} = Sperm_i \times Rate \times EggPos$, including lower-order interactions). We again treated experimental run as a `random' grouping variable, and analyzed the nested model set arising from the interaction of $Run$ with all `fixed' effects (i.e., $\mathbf{Z} = Run \times Sperm_i \times Rate \times EggPos$, including lower-order interactions). We did not drop any of the `fixed-effect' terms in either analysis ($\mathbf{X}$ was identical for all competing models in each analysis).

	Posterior distributions of all model parameters were estimated using Markov Chain Monte Carlo (MCMC) methods implementing a NUTS sampler (\citealt{Stan2016}). For all models in both analyses we initiated $3$ MCMC chains of $2,000$ steps, each with a $1,000$ step warmup, yielding a total of $3,000$ samples (i.e., $(2,000 - 1,000) \times 3 = 3,000$). Chains were monitored for sampling abnormalities and considered to have reached convergence when all parameter estimates yielded a scale reduction statistic of $\hat{R} < 1.01$ (\citealt{GelmanRubin1992}).

	For both analyses, we implemented a fully Bayesian model selection procedure using leave-one-out cross-validation using the R package \textit{loo} v.0.1.6 (PSIS-LOO; \citealt{Vehtari2016}). For each analysis, we compared the fit of all competing models against one another (i.e., all pairwise model comparisons) using the expected log pointwise predictive density ($\widehat{\textit{elpd}}_{\textit{loo}}$), calculated by evaluating the log-likelihood at each of the posterior draws of the parameter values (\citealt{HootenHobbs2015,Vehtari2016}). For each pairwise model comparison, we calculated p-values for the pairwise differences in $\widehat{\textit{elpd}}_{\textit{loo}}$ using standard errors (s.e.) calculated as described in Eq(24) of \citet{Vehtari2016}, and a normal probability density function. Although this method of calculating s.e.'s is most reliable for data sets with many observations (\citealt{Vehtari2016}), we obtained similar results using Bayesian bootstrapping methods to estimate either the s.e.'s, or to sample directly from the distribution of pointwise $\widehat{\textit{elpd}}_{\textit{loo}}$ differences. For our final analyses, we selected the model with the fewest parameters that also returned a pairwise difference in prediction accuracy that was not significantly different from the best-fitting model ($\Pr > 0.05$). All statistical analyses were performed in R v.3.2.2 (\citealt{R2016}). All data and code necessary to reproduce the analyses are available at \url{https://github.com/colin-olito/Fertilization}.

\section*{Results}

	\subsection*{Flume experiments}

	The probability of successful fertilization increased with the amount of sperm released down the flume in Exp.~1 (Fig.~\ref{fig:fertPlots}; $\beta_{Sperm} = 0.819$, HPD interval = $[0.736,0.908]$). The maximum fertilization success acheived was $0.87$, and rates of polyspermy were very low for all runs ($1-2\%$), suggesting that our models adequately estimate the probability of successful fertilization for these data, and more complex fertilization kinetics models are unnecessary. The most parsimonious model included only a single random intercepts term for experimental run ($\mathbf{Z} = Run$; see Supplementary Table S1 for all model comparisons). The $Run$ specific intercepts exhibited only moderate variation ($\sigma_{\gamma} = 0.765$), indicating that the flume was generating repeatable results.

	The most parsimonious model from the analysis of Exp.~2 included several interactions between experimental $Run$ and `fixed' effects ($\mathbf{Z} = Run + Run \times Sperm_i + Run \times Rate + Run \times EggPos + Run \times Rate \times EggPos$; see supplementary Table S2 for all model comparisons). We detected a strong $Sperm_i \times Rate$ interaction, such that the probability of successful fertilization increased faster with the amount of sperm released for the \textit{Fast} release strategy than the \textit{Slow} release strategy (Table~\ref{Table:ModelResults}). We also detected a marginally significant difference in the slope for the $25$ cm and $75$ cm egg patches for the \textit{Fast} release treatment, with fertilization success in the distal egg patch ($75$ cm) increasing faster with the amount of sperm released (Table~\ref{Table:ModelResults}; see also Supplementary Figure~S1).

	Using the results from the analysis of Exp.~2, we calcuated the per-gamete increase in the probability of successful fertilization for the \textit{Fast} and \textit{Slow} release strategies (Fig.~\ref{fig:perGamete}). Per-gamete fertilization rate decreased dramatically with the total amount of sperm released for both strategies, but the decrease was slower for the \textit{Fast} release rate strategy (Table~\ref{Table:ModelResults}). Simple contrasts between the \textit{Fast} and \textit{Slow} strategy prediction lines showed that when small amounts of sperm are released, the per-gamete fertilization rate is significantly higher for the \textit{Slow} strategy than the \textit{Fast} strategy (see Supplementary Figure~S2). Conversely, the \textit{Fast} release strategy acheives higher per-gamete fertilization rates when large amounts of sperm are released.

	\subsection*{Lab spawnings}

	The mean duration of lab-induced male spawning phenologies was $5$:$52.2 \pm 20$ sec.~(mean $\pm$ s.e.). This spawning duration was longer than the time taken to release sperm in the \textit{Slow} release strategy for Exp.~1 ($\approx$4 min), suggesting that our results for the \textit{Slow} strategy in Exp.~2 may be conservative relative to induced male spawning phenologies.


\section*{Discussion}

Since the development of SCT in the 1970's and 1980's, observed patterns of male reproductive investment and allocation in diverse taxa have been interpreted almost exclusively as adaptive strategies in response to sperm competition (\citealt{Parker1972,Parker1982,Wedell2002}). However, we found that for broadcast spawners, many of the same empirical patterns predicted by SCT may also be expected to evolve in response to selection from the external environment during the process of gamete dispersal; regardless of whether sperm competition is present or not. Two main outcomes emerge from our results which we discuss in detail below: (1) an interaction between the amount and rate at which sperm is released, and (2) a dramatic cost, in terms of per-gamete fertilization success, of releasing many sperm.

We found evidence that size- or allocation-dependent sperm release strategies may evolve if males are constrained to a single ejaculate size. If male broadcast spawners are unable to facultatively adjust the amount of sperm they release, our results suggest that small males, or males releasing small ejaculates, should acheive higher fertilization success if they release sperm slowly. The opposite is true for large males, or males releasing large ejaculates. Although there is some evidence that individuals can faculatively adjust the rate of gamete release (\citealt{Marshall2004}), we are aware of no difinitive tests in broadcast spawners of whether males adjust ejaculate sizes at spawning. However, male broadcast spawners often participate in multiple spawning events, and only partially expend their gametic resources in single events (\hl{XXX, Levitan...}). It has also been demonstrated in two species of scallop that males of different sizes release similar amounts of sperm, despite the fact that larger males could have released more (\citealt{Styan2003}). However, these spawnings were chemically induced, and so provide only indirect evidence for faculatative adjustment my large males. 

An interesting corollary to our result, should males be constrained to a single ejaculate size, is the potential for concordant or conflicting selection from sperm dispersal and sperm competition, depending on the size of a male's ejaculate. SCT models for broadcast spawners predict that slow sperm release rates should be favoured under many conditions when sperm competition is present (\citealt{Olito2015,Olito2017}). Our experimental results, however, suggest that selection during sperm dispersal should favour fast release strategies for males releasing large ejaculates. Thus, while males releasing small ejaculates may experience concordant selection from these two processes, large males may experience conflicting selection.

In addition to the interaction between ejaculate size and release rate, we found a massive per-gamete cost associated with releasing many sperm. Taken together, these results suggests that if male broadcast spawners can facultatively adjust their size of their ejaculates, or if they can release sperm over multiple spawning events, the optimal spawning phenology will always be to release as little sperm as possible, as slowly as possible, given the window of time during which females are releasing eggs. Crucially, our results show that this cost is incurred regardless of whether sperm competition is present or not. This is a key result because it suggests that the widely documented patterns of spawning phenology in broadcast spawners, such as high population variances and slow individual release rates, are overdetermined. Moreover, our results suggest that the strength of selection from the external environment during sperm dispersal is likely to be strong even under favourable conditions (e.g., under laminar flow in a flume), while selection from sperm competition is strongly density-dependent. Sperm competition may exaggerate selection for small ejaculates and slow sperm release rates, particularly at high population densities, but selection during sperm dispersal is likely the more consistent force driving the evolution of reproductive phenotypes in broadcast spawners.


% \hl{need to make the point that if males are constrained to specific ejaculate size for a single mating event... males releasing lots of sperm should do so quickly.  BUT... the massive per-gamete costs of releasing lots of sperm suggest that if males can facultatively adjust the amount of sperm they release, or can release sperm over multiple spawning events, they should always release less sperm at a slow rate... regardless of whether sperm competition is present or not. In fact, SC should only exaggerate this effect.}


% Trash
% We also found that the per-gamete fertilization rate for males decreases dramatically with the total amount of sperm released - highlighting the massive investment required by males to overcome sperm wastage. Interesting implications -- predictions from sperm competition agree with our findings for small ejaculates, but in opposition to our findings for large ejaculates... suggesting that small males, or males investing less in sperm production, may experience concordant selection to release sperm slowly. In contrast, large males, or males investing heavily in sperm production may experience opposing selection pressures -- with sperm competition favouring slow release strategies, despite the fact that faster release strategies should result in higher fertilization success in a non-competitive context.



%%%%%%%%%%%%%%%%%%%%%%%%%%%%
\section*{Acknowledgments}
C.O. thanks C.~Venables and C.~Y. Chang for fruitful discussion and feedback, and the MEEG lab for assistance with the flume. C.O. and D.J.M. designed the study, C.O. performed the experiments and analyses, and wrote the manuscript with assistance from D.J.M. This work was funded by a Monash University Dean's International Postgraduate Student Scholarship to C.O., an Australian Research Council grants to D.J.M, and by the Monash University Centre for Geometric Biology.

\newpage{}



%%%%%%%%%%%%%%%%%%%%%%%%%%%%%%%%%%%%%%%%%%%%%%%%%%%%%%%%%%%%%%%%%%
	 \section*{Tables}
	 \renewcommand{\thetable}{\arabic{table}}
	 \setcounter{table}{0}

	\begin{table}[!ht]
	\caption{Parameter estimates of interest and \textit{a priori} contrasts for Exp.~2}
	\label{Table:ModelResults}
	\centering
	\begin{tabular}{l c c c c} 
	\hline
	Parameter & mean & 95\% H.P.D.~interval & 95\% Credible Interval \\
	\hline
	$\beta_{Sperm_i:Slow}$      & $0.705$ & $(0.626,0.781)$ & $(0.623,0.779)$ \\
	$\beta_{Sperm_i:Fast}$      & $0.609$ & $(0.518,0.699)$ & $(0.515,0.698)$ \\
	$\beta_{Sperm_i:Fast:25cm}$ & $0.690$ & $(0.658,0.666)$ & $(0.604,0.767)$ \\
	$\beta_{Sperm_i:Fast:75cm}$ & $0.720$ & $(0.687,0.694)$ & $(0.638,0.792)$ \\
	$\beta_{Sperm_i:Slow:25cm}$ & $0.617$ & $(0.579,0.587)$ & $(0.525,0.705)$ \\
	$\beta_{Sperm_i:Slow:75cm}$ & $0.600$ & $(0.559,0.568)$ & $(0.496,0.695)$ \\
	\hline
	\textit{a priori} contrast & mean & 95\% H.P.D.~interval & $\Pr > 0$ & \\
	\hline
	$\beta_{Sperm_i:Fast}      - \beta_{Sperm_i:Slow}$      & $ 0.096$ & $( 0.059,0.133)$ & $\mathbf{0.000}$  \\
	$\beta_{Sperm_i:Fast:25cm} - \beta_{Sperm_i:Fast:75cm}$ & $-0.029$ & $(-0.067,0.004)$ & $0.056^{\dagger}$ \\
	$\beta_{Sperm_i:Slow:25cm} - \beta_{Sperm_i:Slow:75cm}$ & $ 0.017$ & $(0.034,0.076)$  & $0.743$           \\
	$\beta_{Sperm_i:Fast:25cm} - \beta_{Sperm_i:Slow:25cm}$ & $ 0.073$ & $(0.035,0.106)$  & $\mathbf{1.000}$  \\
	$\beta_{Sperm_i:Fast:75cm} - \beta_{Sperm_i:Slow:75cm}$ & $ 0.119$ & $(0.050,0.192)$  & $\mathbf{1.000}$  \\
	\hline
	\end{tabular}
	\bigskip{}
	\end{table}



%%%%%%%%%%%%%%%%%%%%%%%%%%%%%%%%%%%%%%%%%%%%%%%%%%%%%%%%%%%%%%%%%%
\section*{Figures}
%%%%%%%%%%%%%%%%%%%%%%%%%%%%%%%%%%%%%%%%%%%%%%%%%%%%%%%%%%%%%%%%%%
 
\begin{figure}[!ht] 
\includegraphics[scale=0.39]{/N_invest_Regression}
\caption{Fertilization success as a function of the amount of sperm released for both flume experiments. Panel A shows run-specific regression (grey lines) in addition to the overall regression (black line). Panel B shows the fertilization success adjusted to account for among-run variation (see methods). Credibility intervals have been omitted for clarity. \hl{\ldots}}
\label{fig:fertPlots}
\end{figure}
\newpage{}


\begin{figure}[!ht] 
\includegraphics[scale=0.39]{/perGametePlot}
\caption{Results from 2$^{nd}$ flume experiment showing adjusted per-gamete fertilization success as a function of the amount of sperm released for the fast and slow release treatments. Partial regression lines are plotted (solid lines) with $80\%$ credibility intervals (transparent bands). Data points are adjusted to account for among-run variation (see methods).  \hl{\ldots}}
\label{fig:perGamete}
\end{figure}

\newpage{}






%%%%%%%%%%%%%%%%%%%%%%%%%%%%%%%%%%%%%%%%%%%%%%%%%%%%%%%%%%%%%%%%%%
% Bibliography
%%%%%%%%%%%%%%%%%%%%%%%%%%%%%%%%%%%%%%%%%%%%%%%%%%%%%%%%%%%%%%%%%%

\begin{thebibliography}{}

\bibitem[{Bateman(1948)Bateman}]{Bateman1948}
Bateman, D. 1948.
\newblock Intra-sexual selection in \textit{Drosophila}.
\newblock Heredity 2:349--368.

\bibitem[{Bode and Marshall(2007)Bode and Marshall}]{BodeMarshall2007}
Bode, M. and D.~J. Marshall. 2007.
\newblock The quick and the dead? sperm competition and sexual conflict in the sea.
\newblock Evolution 61:2693--2700.

\bibitem[{Denny and Shibata(1989)Denny and Shibata}]{DennyShib1989}
Denny, M.~W. and M.~F. Shibata. 1989.
\newblock Consequences of surf-zone turbulence for settlement and external fertilization.
\newblock American Naturalist 134:859--889.

\bibitem[{Gelman and Rubin(1992)Gelman and Rubin}]{GelmanRubin1992}
Gelman, A. and D.~B. Rubin. 1992.
\newblock Inference from iterative simulation using multiple sequences.
\newblock Statistical Science 7:457--472.

\bibitem[{Hooten and Hobbs(2015)Hooten and Hobbs}]{HootenHobbs2015}
Hooten, M.~B. and N.~T. Hobbs. 2015.
\newblock A guide to Bayesian model selection for ecologists.
\newblock Ecol.~Mono. 85:3--28.

\bibitem[{Kupriyanova and Havenhand(2013)Kupriyanova and Havenhand}]{Kupriyanova2013}
Kupriyanova, E.~K. and J.~N. Havenhand. 2013.
\newblock Effects of temperature on sperm swimming behaviour, respiration, and fertilization success in the serpulid polychaete, \textit{Galeolaria caespitosa} (Annelida: Serpulidae).
\newblock Inv.~Repro.~Devel. 48:7--17.

\bibitem[{Levitan(1991)Levitan}]{Levitan1991}
Levitan, D. 1991.
\newblock Influence of body size and population density on fertilization success and reproductive output in a free-spawning invertebrate.
\newblock Biological Bulletin 181:261--268.

\bibitem[{Levitan(1998)Levitan}]{Levitan1998}
Levitan, D. 1998.
\newblock Does Bateman's principle apply to broadcast-spawning organisms? Egg traits influence in situ fertilization rates among congeneric sea urchins.
\newblock Evolution 52:1043--1056.

\bibitem[{Levitan(2002)Levitan}]{Levitan2002}
Levitan, D. 2002.
\newblock Density-dependent selection on gamete traits in three congeneric sea urchins.
\newblock Ecology 83:464--479.

\bibitem[{Levitan(2004)Levitan}]{Levitan2004}
Levitan, D. 2004.
\newblock Density-dependent sexual selection in external fertilizers: variances in male and female fertilization success along the continuum from sperm limitation to sexual conflict in the sea urchin \textit{Strongylocentrotus franciscanus}.
\newblock American Naturalist 164:298--309.

\bibitem[{Levitan(2005)Levitan}]{Levitan2005}
Levitan, D. 2005.
\newblock Sex-specific spawning behaviour and its consequences in an external fertilizer.
\newblock American Naturalist 165:682--694.

\bibitem[{Lotterhos and Levitan(2011)Lotterhos and Levitan}]{LotterhosLevitan2011}
Lotterhos, K.~E. and D. Levitan. 2011.
\newblock Gamete release and spawning behaviour in broadcast spawning marine invertebrates.
\newblock Pages 99--120 \emph{in} J.~L. Leonard and A. C\'{o}rdoba-Aguilar, eds. The evolution of primary sexual characters in animals. Oxford University Press, Oxford, UK.

\bibitem[{Marshall(2002)Marshall}]{Marshall2002}
Marshall, D.~J. 2002.
\newblock \textit{In situ} measures of spawning synchrony and fertilization success in an intertidal, free-spawning invertebrate.
\newblock Marine Ecology Progress Series 236:113-119.

\bibitem[{Marshall and Bolton(2007)Marshall and Bolton}]{MarshallBolton2007}
Marshall, D.~J., and T.~F. Bolton. 2007.
\newblock Sperm release strategies in marine broadcast spawners: the costs of releasing sperm quickly.
\newblock J.~Exp.~Biol. 210:3720--3727.

\bibitem[{Marshall and Evans(2005a)Marshall and Evans}]{MarshallEvans2005a}
Marshall, D.~J. and J.~P. Evans. 2005.
\newblock The benefits of polyandry in the free-spawning polychaete \textit{Galeolaria caespitosa}.
\newblock J.~Evol.~Biol. 18:735--741.

\bibitem[{Marshall and Evans(2005b)Marshall and Evans}]{MarshallEvans2005b}
Marshall, D.~J. and J.~P. Evans. 2005.
\newblock Does egg competition occur in marine broadcast spawners? 
\newblock J.~Evol.~Biol. 18:1244-1252.

\bibitem[{Marshall et al.(2004)Marshall et al.}]{Marshall2004}
Marshall, D.~J., D. Semmens, and C. Cook. 2004.
\newblock Consequences of spawning at low tide: limited gamete dispersal for a rockpool anemone.
\newblock Mar.~Ecol.~Prog.~Ser. 266:135--142.

\bibitem[{McEuan(1988)McEuan}]{McEuan1988}
McEuan, F.~S. 1988.
\newblock Spawning behaviors of northeast Pacific sea cucumbers (Holothuroidea: Ecinodermata).
\newblock Marine Biology 98:565--585.

\bibitem[{Monro and Marshall(2015)Monro and Marshall}]{MonroMarshall2015}
Monro, K. and D.~J. Marshall. 2015.
\newblock The biogeography of fertilisation mode in the sea.
\newblock Global Ecology and Biogeography. DOI: 10.1111/geb.12358.

\bibitem[{Olito et~al.(2015)Olito et~al.}]{Olito2015}
Olito, C., M. Bode, and D.~J. Marshall. 2015.
\newblock Evolutionary consequences of fertilization mode for reproductive phenology and asynchrony.
\newblock Mar. Ecol. Prog. Ser. 537:23--38.

\bibitem[{Olito et~al.(2017)Olito et~al.}]{Olito2017}
Olito, C., D.~J. Marshall, and T. Connallon. 2017.
\newblock The Evolution of Reproductive Phenology in Broadcast Spawners and the Maintenance of Sexually Antagonistic Polymorphism.
\newblock American Naturalist. 189: DOI:10.1086/690010.

\bibitem[{Parker(1982)Parker}]{Parker1982}
Parker, G.~A. 1982.
\newblock Why are there so many tiny sperm? sperm competition and the maintenance of two sexes.
\newblock Journal of Theoretical Biology 96:281--294.

\bibitem[{Parker et~al.(1972)Parker, Baker, and Smith}]{Parker1972}
Parker, G.~A., R.~R. Baker, and V.~G.~F. Smith. 1972.
\newblock The origin and evolution of gamete dimorphism and the male-female phenomenon.
\newblock Journal of Theoretical Biology 36:529--553.

\bibitem[{R~Core~Team(2016)R~Core~Team}]{R2016}
R Core Team. 2016.
\newblock R: A language and environment for statistical computing. 
\newblock R Foundation for Statistical Computing, Vienna, Austria. URL https://R-project.org/.

\bibitem[{Rouse and Fitzhugh(1994)Rouse and Fitzhugh}]{RouFitz1994}
Rouse, G. and K. Fitzhugh. 1994.
\newblock Broadcasting fables: is external fertilization really primitive? Sex, size, and larvae in sabellid polychaetes.
\newblock Zoologica Scripta 23:271--312.

\bibitem[{Rstan(2016)Rstan}]{Stan2016}
Stan Development Team. 2016.
\newblock RStan: the R interface to Stan.
\newblock R package version 2.14.1. URL http://mc-stan.org

\bibitem[{Styan and Butler(2003)Styan and Butler}]{Styan2003}
Styan, C.~A. and A.~J. Butler. 2003.
\newblock Scallop size does not predict amount or rate of induced sperm release.
\newblock Mar.~Fresh.~Behav.~Phys. 36:59--65.

\bibitem[{Vehtari et~al.(2016)Vehtari et~al.}]{Vehtari2016}
Vehtari, A., A. Gelman, and J. Gabry. 2016.
\newblock Practical Bayesian model evaluation using leave-one-out cross-validation and WAIC.
\newblock Statistical Computing. DOI: 10.1007/s11222-016-9696-4.

\bibitem[{Wedell et~al.(2002)Wedell, Gage, and Parker}]{Wedell2002}
Wedell, N., M.~J.~G. Gage, and G.~A. Parker. 2002.
\newblock Sperm competition, male prudence, and sperm-limited females.
\newblock Trends in Ecology and Evolution 17:313--320.

\bibitem[{Yund and Meidel(2003)Yund and Meidel}]{YundMeidel2003}
Yund, P.~O. and S.~K. Meidel. 2003.
\newblock Sea urchin spawning in benthic boundary layers: are eggs fertilized before advecting away from females?
\newblock Limnol.~Oceanogr 48:795-801.

\end{thebibliography}

\newpage{}






\end{document}
