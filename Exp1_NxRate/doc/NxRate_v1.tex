%%%%%%%%%%%%%%%%%%%%%%%%%%%%%
%Preamble
\documentclass{article}


%Dependencies
\usepackage[left]{lineno}
\usepackage{titlesec}
\usepackage{amsmath}
\usepackage{amsfonts}
\usepackage{amssymb}
\usepackage{color,soul}
\usepackage{ogonek}


% Packages from Am. Nat. Template
\usepackage[sc]{mathpazo} %Like Palatino with extensive math support
\usepackage{fullpage}
\usepackage[authoryear,sectionbib,sort]{natbib}
\linespread{1.7}
\usepackage{lineno}

% Graphics
\usepackage{graphicx}
\graphicspath{{../output/figs}.pdf}
% New commands: fonts
%\newcommand{\code}{\fontfamily{pcr}\selectfont}
%\newcommand*\chem[1]{\ensuremath{\mathrm{#1}}}

 
%%%%%%%%%%%%%%%%%%%%%%%%%%%%%
% Title Page

\title{Allocation- and size-dependent selection on sperm release phenologies in broadcast spawners}
\author{Colin Olito$^{\ast}$ and Dustin J.~Marshall}
\date{\today}

\begin{document}
\maketitle


\noindent{} Centre for Geometric Biology, School of Biological Sciences, Monash University, Victoria 3800, Australia.
\noindent{} $^\ast$ Corresponding author e-mail: colin.olito@gmail.com

\bigskip

\noindent{} \textit{Keywords}: xxx, ..., xxx

\bigskip

\noindent{} \textit{Manuscript type}: Brief Communication

\bigskip


% Set line number options
\linenumbers
\modulolinenumbers[1]
\renewcommand\linenumberfont{\normalfont\small}

%%%%%%%%%%%%%%%%%%%%%%%%%%%%%
% Main Text

\newpage{}
\section*{Abstract}

\noindent{} \hl{...}

\section*{Introduction}

\begin{enumerate}
	\item Our expectations and predictions regarding the outcome of sexual selection are heavily influenced by the idea that males compete strongly for the fertilization of females' eggs (CITE: Bateman, Parker). \hl{leverage fact that SCT started with external fertilization ... where the biology started, and excludes complicating factors such as cryptic female choice.}
		\begin{enumerate}
			\item Indeed, sperm competition theory represents the largest and most influential body of theory explaining observed patterns of male reproductive investment and allocation. 
			\item Classic predictions from SCT include increased ejaculate investment with competition among species, and facultative adjustment of ejaculate allocation within species to better compete for fertilizations.
			\item While the consequences of sperm competition for the evolution of male reproductive phenotypes are well studied in internally fertilizing taxa, and some externally fertilizing fish species, it is unclear how relevant these predictions are for the archetypal external fertilizer: broadcast spawners (\citealt{BodeMarshall2007,Olito2015}). 
				\begin{enumerate}
					\item This is despite the fact that broadcast spawners constitute more than 50\% of marine invertebrate biodiversity worldwide (\citealt{MonroMarshall2015}), represent the ancestral mode of reproduction (), and have long been used as a model for sperm competition theory and the evolution of anisogamy (Parker 1982; Parker et al. 1972).
				\end{enumerate}
		\end{enumerate}

	\item On the surface, broadcast spawning species appear well suited to predictions from sperm competition theory.
		\begin{enumerate}
			\item Gametes are released into the water column by multiple individuals, often in dense aggregations.
			\item Consequently, polyandry is extremely common (CITE Levitan urchins and corals, Johnson \& Yund (2007) for brooders), and sperm competition is probably intense for many broadcast spawning species.
		\end{enumerate}

	\item However, several well documeted empirical patterns in broadcast spawning taxa do not appear consistent with SCT predictions.
		\begin{enumerate}
			\item Many species exhibit surprisingly long population spawning phenologies characterized by slow gamete release rates by individuals.
			\item Large males do not necessarily release more sperm than small males, despite having the resources available to do so (\citealt{Styan2003})
			\item Protandry is also common among broadcast spawners, with males releasing sperm earlier for for longer than females do eggs (\citealt{McEuan1988, LotterhosLevitan2011, Levitan2005, MarshallBolton2007}).
		\end{enumerate}

	\item \hl{Need to be deft, both here and in the discussion, in explaining why we do not do this in a sperm-competitive context, despite the fact that ALL the theory is about SC. The play is: absence of sperm competition is the baseline, selection from environment is expected to be VERY strong for broadcasters, and our results suggest that SC doesn't change the predictions that much.}

	\item 

	\item These reproductive phenotypes are particularly counterintuitive from the perspective of sperm competition theory because males releasing sperm faster should enjoy a fitness advantage by acheiving numerical dominance over their competitors.
		\begin{enumerate}
			\item Further confounding SCT predictions is the fact that broadcast spawners generally have massive gonads relative to even the most extravagant internal fertilizers (\hl{NEED TO PUT A NUMBER AND SOME CITATIONS ON THIS}).
			\item Apparent disagreements between SCT and observed broadcast spawning strategies may be a consequence strong selection 	from the external environment on gametes during dispersal.
			\begin{enumerate}
				\item The process of gamete dispersal is characterized by rapid diffusion, dependent on local hydrodynamic conditions, resulting in severe wastage of male ejaculates that never encounter eggs in the water column.
				\item Fertilization success is ultimately determined by local gamete concentrations... sperm limitation \(\leftrightarrow\) polyspermy.
				\item Ultimately, males must optimize local sperm concentrations in order to acheive higher fertilization success.
				\item Crucially, males are able to directly manipulate local sperm concentrations by adjusting their sperm release phenologies. Thus, spawning phenologies represent an important target of selection during the process of gamete dispersal.
			\end{enumerate}	
		\end{enumerate}

	\item The critical question is what are the fitness consequences of different male spawning phenologies?
		\begin{enumerate}
			\item Previous theoretical and empirical studieshave investigated the fertilization success of different male broadcast spawning strategies in a sperm competitive context.
			\begin{enumerate}
				\item A crucial prediction from these models is that sperm competition should often select for males that release sperm more slowly than females (\citealt{BodeMarshall2007, Olito2015, Olito2017}).
				\item There have been relatively few empirical studies that measure ...blah blah... sperm competitive trials (reviewed in \citealt{LotterhosLevitan2011}), but suggest that males who release sperm slowly, and earlier than females enjoy higher fertilization success than rival males who release sperm faster, and in closer synchrony with females.
			\end{enumerate}
			\item However, very few studies have examined selection from the external environment during sperm dispersal, outside of a sperm competitive context.
			\item \citet{BodeMarshall2007} is one of the few we are aware of that directly manipulated male spawning strategies.
			\item tehy found that males releasing sperm slowly enjoyed higher fertilziation success at two downstream egg patches in a flume experiment than did fast-releasing males... however they did not manipulate X, Y, Z.
			\item Critical gap in our understanding: Fertilization success of more diverse reproductive phenotypes. In particular, how should sperm release rate covary with the energetic investment males are making in their ejaculate?
		\end{enumerate}
	\item To address this question empirically, we conducted two flow-through flume experiments to simulate different male reproductive phenotypes using gametes collected form the broadcast spawning marine invertebrate \textit{Galeolaria saespitosa}. We experimentally manipulated the total amount, and the rate at which sperm was released, and analyzed the resulting fertilization success of two downstream egg patches.
		\begin{enumerate}
			\item We found that the sperm release rate that resulted in higer fertilization success depended critically on the total amount of sperm being released. Slow-release strategies resulted in higher fertilization when ejaculates were relatively small, while fast-release strategies resulted in higher fertilization rates when ejaculates were relatively large. We also found that the per-gamete fertilization rate for males decreases dramatically with the total amount of sperm released - highlighting the massive investment required by males to overcome sperm wastage... 
			\item Interesting implications -- predictions from sperm competition agree with our findings for small ejaculates, but in opposition to our findings for large ejaculates... suggesting that small males, or males investing less in sperm production, may experience concordant selection to release sperm slowly. In contrast, large males, or males investing heavily in sperm production may experience opposing selection pressures -- with sperm competition favouring slow release strategies, despite the fact that faster release strategies should result in higher fertilization success in a non-competitive context.
		\end{enumerate}

\end{enumerate}



\section*{Methods}

	\subsection*{Study species}a
	We studied spawning phenologies and the fertilization success of different male spawning strategies using \textit{Galeolaria caespitosa} Savigny 1818. \textit{G.~caespitosa} is a dioecious sessile polychaete worm common in the intertidal zone throughout southeast Australia. For this study, all individuals were collected from pier pilings at the Royal Brighton Yacht Squadron, in Port Philip Bay, Victoria. For the flume experiments, we used standard protocols to collect gametes (\citealt{MarshallEvans2005a,MarshallEvans2005b}). \textit{G.~caespitosa} sperm are active immediately upon release, and can remain viable for several hours after spawning (\citealt{Kupriyanova2013}). To minimize any effects of sperm aging on egg fertilization success, we ensured that all gametes used in the flume experiments had been freshly spawned within $45$~min of experimental runs.

	\subsection*{Lab spawnings}

	To estimate the duration of individual spawning phenologies in the wild, we documented induced, non-traumatic, spawnings in a laboratory setting. A large \textit{G.~caespitosa} colony fragment was collected on \date{16/11/16}, and stored overnight in a controlled temperature room at the ambient temperature of Port Philip Bay ($17.5^o$C). The large colony fragment was then sub-divided into small fragments of approx.~$50-100$ individuals that were then induced to spawn by heat shock in a shallow sea-water bath. Heat shocking was performed by warming the water bath from $17.5^o$ to approximately $26^o$C in approx.~$10$~min. Individual spawning males were marked and monitored and the duration of their spawning timed. Water in the bath was kept gently agitated, and spawning individuals were periodically flushed with seawater using a 1cc syringe. Individual males' spawning phenology was considered complete when no more ejaculate could be seen leaking from the tube with gentle flushing. Approximately $8-10$ small colony fragments were monitored at once in an experimental run, and we performed $5$ experimental runs resulting in $26$ documented individual spawning phenologies. 

	\subsection*{Flume experiments}
	We performed two flume experiments (experiments 1 and 2 hereafter) to examine the effect of the amount and rate of sperm released in simulated broadcast spawning events on fertilization success at stationary egg patches sitting downstream. The flume was plexiglass, with internal dimensions of $2.4 \times 0.96 \times 0.1$m (L$\times$W$\times$D), divided into $6$ parallel lanes, each $0.15$m wide which were fed from a common head tank. When in use, the flume was filled with seawater to a depth of $5$cm. Laminar flow in each lane was maintained by $10$cm collimators made of drinking straws located $0.25$m from the head of each lane (\citealt{YundMeidel2003}). Filtered seawater sourced from the Mornington Peninsula, Port Philip Bay, was gravity fed into a $20$-L constant head tank, which in turn fed the common head for the flume, ensuring constant flow rates during experimental runs.

	\subsection*{Analyses}



\section*{Results}

	\subsection*{Flume experiment}

	Mean spawning duration of $5$:$52.2 \pm 20$ sec.~(mean $\pm$ s.e.).

	\subsection*{Analyses}


\section*{Discussion}

\hl{need to make the point that if males are constrained to specific ejaculate size for a single mating event... males releasing lots of sperm should do so quickly.  BUT... the massive per-gamete costs of releasing lots of sperm suggest that if males can facultatively adjust the amount of sperm they release, or can release sperm over multiple spawning events, males should always release less sperm, and slowly... regardless of whether sperm competition is present or not. In fact, SC should only exaggerate this effect.}



\section*{Acknowledgments}
C.O. thanks C.~Venables, C.~Y. Chang, for fruitful discussion and feedback, and the MEEG lab for assistance with the flume. C.O. and D.J.M conceived the study, C.O. performed the experiments and analyses, and wrote the manuscript with assistance from D.J.M. This work was funded by a Monash University Dean's International Postgraduate Student Scholarship to C.O., an Australian Research Council grants to D.J.M, and by the Monash University Centre for Geometric Biology.

\newpage{}




%%%%%%%%%%%%%%%%%%%%%%%%%%%%%%%%%%%%%%%%%%%%%%%%%%%%%%%%%%%%%%%%%%
% \section*{Tables}
% \renewcommand{\thetable}{\arabic{table}}
% \setcounter{table}{0}

%\begin{table}[h!]
%\caption{Model comparison results?}
%\label{Table:???}
%\centering
%\begin{tabular}{l c c c c} \hline
%Haplotype & $y_1 = AB$ & $y_2 = Ab$ & $y_3 = aB$ & $y_4 = ab$ \\
%\hline
%Females & & & & \\
%$x_1 = AB$ & $(1-s_f)^2$ & $(1 - s_f)(1 - h_f s_f)$ & $(1 - s_f)(1 - h_f s_f)$ & $(1 - h_f s_f)^2$ \\
%$x_2 = Ab$ & $(1 - s_f)(1 - h_f s_f)$ & $(1-s_f)$ & $(1 - h_f s_f)^2$ & $(1 - h_f s_f)$ \\
%$x_3 = aB$ & $(1 - s_f)(1 - h_f s_f)$ & $(1 - h_f s_f)^2$ & $(1-s_f)$ & $(1 - h_f s_f)$ \\
%$x_4 = ab$ & $(1 - h_f s_f)^2$ & $(1 - h_f s_f)$ & $(1 - h_f s_f)$ & $1$ \\
%Males & & & & \\
%$x_1 = AB$ & $1$ & $(1 - h_m s_m)$ & $(1 - h_m s_m)$ & $(1 - h_m s_m)^2$ \\
%$x_2 = Ab$ & $(1 - h_m s_m)$ & $(1-s_m)$ & $(1 - h_m s_m)^2$ & $(1 - s_m)(1 - h_m s_m)$ \\
%$x_3 = aB$ & $(1 - h_m s_m)$ & $(1 - h_m s_m)^2$ & $(1-s_m)$ & $(1 - s_m)(1 - h_f s_f)$ \\
%$x_4 = ab$ & $(1 - h_m s_m)^2$ & $(1 - s_m)(1 - h_m s_m)$ & $(1-s_m)(1 - h_m s_m)$ & $(1-s_m)^2$ \\
%\hline
%\end{tabular}
%\bigskip{}
%\end{table}




%%%%%%%%%%%%%%%%%%%%%%%%%%%%%%%%%%%%%%%%%%%%%%%%%%%%%%%%%%%%%%%%%%
\section*{Figures}
%%%%%%%%%%%%%%%%%%%%%%%%%%%%%%%%%%%%%%%%%%%%%%%%%%%%%%%%%%%%%%%%%%
 
\begin{figure}[!ht] 
\includegraphics[scale=0.46]{/fertPlots}
\caption{Fertilization success as a function of the amount of sperm released for both flume experiments. Panel A shows run-specific regression (grey lines) in addition to the overall regression (black line). Panel B shows the fertilization success adjusted to account for among-run variation (see methods). Credibility intervals have been omitted for clarity. \hl{\ldots}}
\label{fig:fertPlots}
\end{figure}
\newpage{}


\begin{figure}[!ht] 
\includegraphics[scale=0.8]{/perGametePlot}
\caption{Results from 2$^{nd}$ flume experiment showing adjusted per-gamete fertilization success as a function of the amount of sperm released for the fast and slow release treatments. Partial regression lines are plotted (solid lines) with $80\%$ credibility intervals (transparent bands). Data points are adjusted to account for among-run variation (see methods).  \hl{\ldots}}
\label{fig:perGamete}
\end{figure}
\newpage{}




%%%%%%%%%%%%%%%%%%%%%%%%%%%%%%%%%%%%%%%%%%%%%%%%%%%%%%%%%%%%%%%%%%
%\subsection*{Supporting information and appendices}
%%%%%%%%%%%%%%%%%%%%%%%%%%%%%%%%%%%%%%%%%%%%%%%%%%%%%%%%%%%%%%%%%%
\renewcommand{\thefigure}{A\arabic{figure}}
\setcounter{figure}{0}

%\begin{figure}[!ht]
%\includegraphics[width=\textwidth]{./figA1}
%\caption{....}
%\label{FigA:figA1.pdf}
%\end{figure}
%\newpage{}




%%%%%%%%%%%%%%%%%%%%%%%%%%%%%%%%%%%%%%%%%%%%%%%%%%%%%%%%%%%%%%%%%%
% Bibliography
%%%%%%%%%%%%%%%%%%%%%%%%%%%%%%%%%%%%%%%%%%%%%%%%%%%%%%%%%%%%%%%%%%

\begin{thebibliography}{}

\bibitem[{Bode and Marshall(2007)Bode and Marshall}]{BodeMarshall2007}
Bode, M. and D.~J Marshall. 2007.
\newblock The quick and the dead? sperm competition and sexual conflict in the sea.
\newblock Evolution 61:2693--2700.

\bibitem[{Kupriyanova and Havenhand(2013)Kupriyanova and Havenhand}]{Kupriyanova2013}
Kupriyanova, E.~K. and J.~N. Havenhand. 2013.
\newblock Effects of temperature on sperm swimming behaviour, respiration, and fertilization success in the serpulid polychaete, \textit{Galeolaria caespitosa} (Annelida: Serpulidae).
\newblock Inv.~Repro.~Devel. 48:7--17.

\bibitem[{Levitan(1998)Levitan}]{Levitan1998}
Levitan, D. 1998.
\newblock Does Bateman's principle apply to broadcast-spawning organisms? Egg traits influence in situ fertilization rates among congeneric sea urchins.
\newblock Evolution 52:1043--1056.

\bibitem[{Levitan(2005)Levitan}]{Levitan2005}
Levitan, D. 2005.
\newblock Sex-specific spawning behaviour and its consequences in an external fertilizer.
\newblock American Naturalist 165:682--694.

\bibitem[{Lotterhos and Levitan(2011)Lotterhos and Levitan}]{LotterhosLevitan2011}
Lotterhos, K.~E. and D. Levitan. 2011.
\newblock Gamete release and spawning behaviour in broadcast spawning marine invertebrates.
\newblock Pages 99--120 \emph{in} J.~L. Leonard and A. C\'{o}rdoba-Aguilar, eds. The evolution of primary sexual characters in animals. Oxford University Press, Oxford, UK.

\bibitem[{Marshall and Bolton(2007)Marshall and Bolton}]{MarshallBolton2007}
Marshall, D.~J., and T.~F. Bolton. 2007.
\newblock Sperm release strategies in marine broadcast spawners: the costs of releasing sperm quickly.
\newblock J.~Exp.~Biol. 210:3720--3727.

\bibitem[{Marshall and Evans(2005a)Marshall and Evans}]{MarshallEvans2005a}
Marshall, D.~J. and J.~P. Evans. 2005.
\newblock The benefits of polyandry in the free-spawning polychaete \textit{Galeolaria caespitosa}.
\newblock J.~Evol.~Biol. 18:735--741.

\bibitem[{Marshall and Evans(2005b)Marshall and Evans}]{MarshallEvans2005b}
Marshall, D.~J. and J.~P. Evans. 2005.
\newblock Does egg competition occur in marine broadcast spawners? 
\newblock J.~Evol.~Biol. 18:1244-1252.

\bibitem[{McEuan(1988)McEuan}]{McEuan1988}
McEuan, F.~S. 1988.
\newblock Spawning behaviors of northeast Pacific sea cucumbers (Holothuroidea: Ecinodermata).
\newblock Marine Biology 98:565--585.

\bibitem[{Yund and Meidel(2003)Yund and Meidel}]{YundMeidel2003}
Yund, P.~O. and S.~K. Meidel. 2003.
\newblock Sea urchin spawning in benthic boundary layers: are eggs fertilized before advecting away from females?
\newblock Limnol.~Oceanogr 48:795-801.

\bibitem[{Monro and Marshall(2015)Monro and Marshall}]{MonroMarshall2015}
Monro, K. and D.~J. Marshall. 2015.
\newblock The biogeography of fertilisation mode in the sea.
\newblock Global Ecology and Biogeography. DOI: 10.1111/geb.12358.

\bibitem[{Olito et~al.(2015)Olito et~al.}]{Olito2015}
Olito, C., M. Bode, and D.~J. Marshall. 2015.
\newblock Evolutionary consequences of fertilization mode for reproductive phenology and asynchrony.
\newblock Mar. Ecol. Prog. Ser. 537:23--38.

\bibitem[{Olito et~al.(2017)Olito et~al.}]{Olito2017}
Olito, C., D.~J. Marshall, and T. Connallon. 2017.
\newblock The Evolution of Reproductive Phenology in Broadcast Spawners and the Maintenance of Sexually Antagonistic Polymorphism.
\newblock American Naturalist. 189: DOI:10.1086/690010.

\bibitem[{Styan and Butler(2003)Styan and Butler}]{Styan2003}
Styan, C.~A. and A.~J. Butler. 2003.
\newblock Scallop size does not predict amount or rate of induced sperm release.
\newblock Mar.~Fresh.~Behav.~Phys. 36:59--65.

\end{thebibliography}

\newpage{}






\end{document}
