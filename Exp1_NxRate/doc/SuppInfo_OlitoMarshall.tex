% Preamble
\documentclass{article}

%Dependencies
\usepackage[left]{lineno}
%\usepackage{indentfirst}
\usepackage{titlesec}
\usepackage{amsmath}
\usepackage{amsfonts}
\usepackage{amssymb}
\usepackage[utf8]{inputenc}
\usepackage{amsmath}
\usepackage{amsfonts}
\usepackage{amssymb}
\usepackage{color,soul}
%\usepackage{times}
\usepackage[sc]{mathpazo} %Like Palatino with extensive math support
\usepackage[authoryear,sectionbib,sort]{natbib}
\usepackage[hidelinks]{hyperref}
\linespread{1.25}
% Default margins are too wide all the way around. I reset them here
\usepackage{fullpage}

% Running headers
\usepackage{fancyhdr}
\setlength{\headheight}{28pt}

% Graphics package
\usepackage{graphicx}
\graphicspath{{../output/figs/}.pdf}

% New commands: fonts
\newcommand{\code}{\fontfamily{pcr}\selectfont}
%\newcommand*\chem[1]{\ensuremath{\mathrm{#1}}}
\newcommand\numberthis{\addtocounter{equation}{1}\tag{\theequation}}

%% Other Options

% Change subsection numbering
\renewcommand\thesubsection{\arabic{subsection})}
\renewcommand\thesubsubsection{}

% Subsubsection Title Formatting
\titleformat{\subsubsection}    
{\normalfont\fontsize{12pt}{17}\itshape}{\thesubsubsection}{12pt}{}


%%%%%%%%%%%%%%%%%%%%%%%%%%%%%%%%%%%%%%%%%%%%
\title{Supporting Information (Appendices A \& B) for: Sperm release strategies in external fertilizers: natural selection augments sexual selection. \textit{Evolution}}

%\author{Colin Olito$^{\ast}$ and Dustin J.~Marshall}

\date{\today}

\begin{document}
\maketitle

%\noindent{} Centre for Geometric Biology, School of Biological Sciences, Monash University, Victoria 3800, Australia.
\bigskip

%\noindent{} $^\ast$ Corresponding author e-mail: \url{colin.olito@gmail.com}
\bigskip

\newpage


%%%%%%%%%%%%%%%%%%%%%%%%%%%%%%%%%%%%%%%%%%%%%%%%
\subsection*{Appendix A: Model selection results}
\renewcommand{\theequation}{S\arabic{equation}}
\setcounter{equation}{0}
\renewcommand{\thefigure}{S\arabic{figure}}
\setcounter{figure}{0}
\renewcommand{\thetable}{S\arabic{table}}
\setcounter{table}{0}


Model selection for the analysis of the \textul{Investment} experiment required a comparison between 3 candidate models with the following 'random effect' $\mathbf{Z}$ matrices:

\begin{align*}
	\mathbf{Z_{m1}} = &Run + Run \times Sperm_i \\
	\mathbf{Z_{m2}} = &Run \\
	\mathbf{Z_{m3}} = &N/A \\
\end{align*}

\noindent{} which yielded the following results: 

\begin{table}[!ht]
\caption{Model Selection for Exp.~1 analysis}
\label{Table:Exp1ModComp}
\centering
\begin{tabular}{l c c c } \hline
Comparison & $\Delta~\widehat{\textit{elpd}}_{\textit{loo}}$ & $s.e.~(\Delta~\widehat{\textit{elpd}}_{\textit{loo}})$ & $\Pr(\Delta~\widehat{\textit{elpd}}_{\textit{loo}} > 0)$ \\
\hline
m1 v.~ \textbf{m2}  & 1.485   & 6.655  & 0.823 \\
m1 v.~ m3 & 134.224 & 30.675 & 0.00  \\
m2 v.~ m3 & 132.739 & 31.572 & 0.00  \\
\hline
\end{tabular}
\bigskip{}
\end{table}

\newpage{}

\noindent Model selection for the \textul{Rate} experiment analysis required a comparison between 25 candidate models (m1 through m25, in order of decreasing $\mathbf{Z}$ matrix complexity; see Supplementary R code for specification of all 'random effect' model matrices $\mathbf{Z_{m1}}$ through $\mathbf{Z_{m25}}$). Here, we present the pairwise comparisons between the best-fitting model (according to $\widehat{\textit{elpd}}_{\textit{loo}}$), and all other candidate models. The most parsimonious model -- the one with the fewest terms whose fit to the data is not significantly different from the best-fitting model -- is highlighted in bold.


\begin{table}[!ht]
\caption{Model Selection results for the \textul{Rate} experiment}
\label{Table:Exp2ModComp}
\centering
\begin{tabular}{l c c c c} \hline
Comparison & $\Delta~\widehat{\textit{elpd}}_{\textit{loo}}$ & $s.e.~(\Delta~\widehat{\textit{elpd}}_{\textit{loo}})$ & $\Pr(\Delta~\widehat{\textit{elpd}}_{\textit{loo}} > 0)$ \\
\hline
m2 v.~ m1  & 0.605   & 2.256  & 0.789 \\
m2 v.~ m3  & 5.643   & 4.260  & 0.185 \\
m2 v.~ \textbf{m12} & 6.045   & 10.612 & \textbf{0.569} \\
m2 v.~ m6  & 6.518   & 7.802  & 0.404 \\
m2 v.~ m4  & 7.750   & 4.292  & 0.071 \\
m2 v.~ m9  & 8.692   & 7.579  & 0.251 \\
m2 v.~ m7  & 9.455   & 7.871  & 0.230 \\
m2 v.~ m10$^\dagger$ & 15.727  & 8.257  & 0.057 \\
m2 v.~ m17 & 36.786  & 14.180 & 0.009 \\
m2 v.~ m11 & 38.400  & 12.588 & 0.002 \\
m2 v.~ m5  & 38.548  & 11.773 & 0.001 \\
m2 v.~ m14 & 38.800  & 12.712 & 0.002 \\
m2 v.~ m8  & 42.059  & 11.996 & 0.000 \\
m2 v.~ m13 & 42.276  & 12.818 & 0.001 \\
m2 v.~ m16 & 43.129  & 13.303 & 0.001 \\
m2 v.~ m19 & 51.253  & 20.688 & 0.013 \\
m2 v.~ m18 & 68.478  & 23.460 & 0.004 \\
m2 v.~ m21 & 78.187  & 27.890 & 0.005 \\
m2 v.~ m15 & 99.122  & 23.416 & 0.000 \\
m2 v.~ m20 & 118.226 & 26.488 & 0.000 \\
m2 v.~ m22 & 121.954 & 28.257 & 0.000 \\
m2 v.~ m23 & 139.700 & 28.561 & 0.000 \\
m2 v.~ m24 & 140.445 & 29.481 & 0.000 \\
m2 v.~ m25 & 322.533 & 68.962 & 0.000 \\
\hline
\end{tabular}
\smallskip{} \\
{\footnotesize Note: Bold indicates the most parsimonious model selected, while daggers \\indicate other competitive candidate models with a similar fit to the data, \\but more parameters.}
\end{table}










\newpage{}

%%%%%%%%%%%%%%%%%%%%%%%%%%%%%%%%%%%%%%%%%%%%%%%%
\subsection*{Appendix B: Supplementary figures}
\renewcommand{\theequation}{S\arabic{equation}}
\setcounter{equation}{0}
\renewcommand{\thefigure}{S\arabic{figure}}
\setcounter{figure}{0}

\begin{figure}[!ht]
\includegraphics[width=\textwidth]{./NxRate_coefContrasts}
\caption{Density plots showing the posterior distributions for \textit{a priori} contrasts of the regression coefficients from the analysis of the \textul{Rate} experiment. A difference of 0 is benchmarked in each panel by a red vertical line.}
\label{fig:coefContr}
\end{figure}
\newpage{}



\begin{figure}[!ht]
\includegraphics[width=\textwidth]{./NxRate_simpleContrasts}
\caption{Density plots showing the posterior distributions for simple contrasts of the predicted difference in fertilization success for the \textit{Fast} and \textit{Slow} release strategies at different values of the $Sperm_i$. Contrasts are calculated at values of $Sperm_i$ corresponding to $-1 \sigma$, the mean ($0$), $+1 \sigma$, and $+2 \sigma$, where $\sigma$ is the empirical standard deviation of the Z-transformed number of sperm released, $Sperm_i$, from the \textul{Rate} experiment ($8.9 \times 10^{7}$). A difference of 0 is benchmarked in each panel by a red vertical line.}
\label{fig:simpContr}
\end{figure}
\newpage{}



%%%%%%%%%%%%%%%%%%%%%%%%%%%%%%%%%%%%%%%%%%%%%%%%
\subsection*{Supplementary R code}
\renewcommand{\theequation}{S\arabic{equation}}
\setcounter{equation}{0}
\renewcommand{\thefigure}{S\arabic{figure}}
\setcounter{figure}{0}
\renewcommand{\thetable}{S\arabic{table}}
\setcounter{table}{0}

See online supplementary material files {\code N\_invest\_FitModels.R}, and {\code NxRate\_FitModels\_Bin.R}.
\newpage
%%%%%%%%%%%%%%%%%%%%%
% Bibliography
%%%%%%%%%%%%%%%%%%%%%

%\begin{thebibliography}{}
%
%
%\bibitem[{Caballero and Hill(1992)Caballero and Hill}]{CaballeroHill1992}
%Caballero, A. and W.~G. Hill. 1992.
%\newblock Effects of partial inbreeding on fixation rates and variation of mutant genes.
%\newblock Genetics 131:493--507.
%
%\bibitem[{Holden(1979)Holden}]{Holden1979}
%Holden, L.~R. 1979.
%\newblock New properties of the two-locus partial selfing model with selection.
%\newblock Genetics 93:217--236.
%
%\bibitem[{Jordan and Connallon(2014)Jordan and Connallon}]{JordanConn2014}
%Jordan, C.~Y. and T. Connallon. 2014.
%\newblock Sexually antagonistic polymorphism in simultaneous hermaphrod-\\ites.
%\newblock Evolution 68:3555--3569.
%
%\bibitem[{Otto and Day(2007)Otto and Day}]{OttoDay2007}
%Otto, S.~P. and T. Day. 2007.
%\newblock A biologist's guide to mathematical modeling in ecology and evolution.
%\newblock Princeton University Press, Princeton, New Jersey, USA.
%
%
%\end{thebibliography}

\end{document}
