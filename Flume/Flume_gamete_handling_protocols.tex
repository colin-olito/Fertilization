%===========================================
% Preamble

\documentclass[12pt]{article}

%**********
%Dependencies
%\usepackage[left]{lineno}
\usepackage{titlesec}
\usepackage{amsmath}
\usepackage{amsfonts}
\usepackage{amssymb}
%\usepackage[utf8]{inputenc}
\usepackage{color,soul}
%\usepackage{setspace}
%\usepackage{times}
\usepackage{times}


% Packages from Am. Nat. Template
%\usepackage[sc]{mathpazo} %Like Palatino with extensive math support
\usepackage{fullpage}
%\usepackage[authoryear,sectionbib,sort]{natbib}
%\linespread{1.7}
%\usepackage[utf8]{inputenc}
%\usepackage{lineno}

%===========================================
%===========================================
% Title Page
\title{Flume Gamete Handling Protocols}

\author{Colin Olito}

\date{\today}

\begin{document}

\maketitle

\newpage{}
%===========================================
% Document

\section{Gamete preparation}
\subsection{Males}

1. Remove males from their tubes, and place into individual petri dishes filled with 3 mL of FSW for spawning. \bigskip

\noindent 2. Draw up concentrated sperm for the Nm males needed for the day, trying to collect ~0.1mL of concentrated sperm solution per male. \bigskip

\noindent 3. Pool sperm, mix thoroughly. \bigskip

\noindent 4. Remove 1 x 0.1mL aliquot for enumeration. \bigskip

a. Perform serial dilution

b. Count sperm w/ Haemocytometer and back-calculate to get the concentration of the original pooled sperm sample. Below is a simple R function for this back-calculation: \bigskip

\[
conc \leftarrow function(x,d)\{mean(x) * (\frac{1}{4E-6})*d\}
\]
where $x$ is a vector of sperm cell counts per 4nL square, and $d$ is the dilution factor(e.g. 1E+4) \bigskip

Usage:  conc(c(x1,x2,x3,...),1e4) \bigskip

\noindent 5. Using the back-calculated concentration, dilute the pooled sperm sample to give a concentration of 1.0E-7 ($M_1 V_1 = M_2 V_2$)

a. This should (!?!?) be in the ballpark of $~10\times$ the original volume of the undiluted pooled sperm solution. But, Marshall and Evans (2005) got 20mL with 10 males. May need to dilute by another $10\times$ to get volumes that are easier to deal with (e.g. 130mL to subdivide rather than 13 if using 13 males). \bigskip

\noindent 6. Sub-divide the now diluted pooled ejaculate according to the day's treatments so the volume used for lane $i$ is: $(N_m-1) \times V_{pooled} \times N_i$


\end{document}



